\chapter{Technologie}

\section{Framework}

Pro vývoj na microcontroller ESP32 se používají hlavně dva frameworky a~to ESP-IDF \cite{ESP-IDF} a Arduino \cite{arduino}, oba jsou pro jazyk C/C++.

\subsection{ESP-IDF}

ESP-IDF, nebo také Espressif IoT Development Framework, je oficiální framework od výrobce ESP32, firmy Espressif Systems \cite{espressif}.
Je psaný pro vývoj v~jazyce C a~C++, samotný je psaný v~jazyce C.
Obsahuje několik úrovní abstrakce od přímé práce s~registry pro uživatelsky přívětivé API.

\begin{minipage}{\linewidth}
\lstinputlisting[language=C++, caption=ESP-IDF]{code/esp-idf.cpp}
\end{minipage}

\subsection{Arduino}

\enlargethispage{5mm}
Arduino je pro začátečníka jednoduší na pochopení a~na práci než ESP-IDF, protože funguje jako úroveň abstrakce nad ESP-IDF.
Jeho obrovskou předností a~zároveň jeho největší limitací je jeho kompatibilita pro množství naprosto rozličných platforem a~architektur sahající až po 8-bitové mikročipy ATtiny.~Bohužel stabilní vývojová větev Arduina pro ESP32 používá zastaralou verzi ESP-IDF, která má některá omezení, kupříkladu nepodporuje C++~17.

\begin{minipage}{\linewidth}
    \lstinputlisting[language=C++, caption=Arduino]{code/arduino.cpp}
\end{minipage}

\subsection{Další frameworky}

Samozřejmě existují i~další frameworky, kupříkladu MicroPython \cite{uPython} a~CircuitPython \cite{circuitPython}, které přivádí jazyk Python na mikrokotrolery, nebo Espruino \cite{espruino}, které dělá to samé pro JavaScript.

\subsection{Výběr}

Pro BlackBox jsem se rozhodl použít přímo ESP-IDF.
K~tomuto rozhodnutí mě vedl fakt, že moje knihovna bude sloužit jako úroveň abstrakce, tudíž Arduino mezivrstva je v~podstatě zbytečná, zároveň tím získám lepší kontrolu nad ESP32.
Protiargumentem by mohlo být množství knihoven dostupných pro Arduino.
Bohužel tyto knihovny nedodržují žádný společný rámec a~většina z~nich není uzpůsobena pro práci na více jádrových procesorech, jakým ESP32 je.\footnote{Nejsou thread safe.}
Zároveň toto umožní zpětnou kompatibilitu s oběma frameworky.

\section{Použité knihovny}

\begin{itemize}
    \item SmartLeds \cite{SmartLeds} --
        Knihovna pro interakci s chytrými led WS2812 pomocí hardwarové periferie RMT na ESP32. % fixme odkazy, zdroje
    \item Eventpp --
        Knihovna pro jednoduchou práci s událostmi.
\end{itemize}