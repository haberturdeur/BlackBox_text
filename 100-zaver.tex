\chapter{Závěr}

Zadaný cíl práce jsem splnil,
navrhl jsem sadu knihoven pro práci s~deskou BlackBox zaměřenou na tři hlavní cíle: % fixme Vojta:"byla naprogramována Trošku bych to "byla" krotil, pokud není, a do krajského/státního nejspíš nebude - mohlo by se porotcům zdát, že kecáš, což nechceme."

\begin{itemize}[noitemsep]
	\item vývoj IoT na stavebnici BlackBox
	\item realizaci outdoorových her a zážitkových akcí
	\item výuku programování zaměřenou na programování mikrokontrolérů, ovládání externích periferií mikrokontrolérů a hobby robotiku
\end{itemize}

K těmto knihovnám jsem také sepsal dokumentaci \cite{dokumentace} v češtině a v angličtině. 

Knihovny jsou jednoduché na použití a~přitom poskytují širokou škálu možností práce s BlackBoxem, od uživatelského přístupu
(stáhnu hru z webu, nahraji do BlackBoxu, vysvětlím účastníkům pravidla a můžu vyrazit do akce) až po pokročilé programování. 

Na této sadě knihoven byla ve školním roce 2019/20 vedena část kroužku robotiky, % fixme název kroužku 
bohužel tento kroužek nemohl být dokončen kvůli globální pandemii.
Kvůli pandemii také nebylo zatím možné použít BlackBox na žádné velké venkovní hře, vzhledem k~nemožnosti jejich konání.

V~práci na BlackBox budu i~nadále pokračovat, a~to nejen na rozšiřování o~periferie chystané pro verzi 2.0 hardwarové desky, 
která bude přidávat hlavně GPS, GPRS a~sekundární procesor.
Ale také vytvářením dalších softwarových pomůcek jako například grafický editor her a~generátor her.

Knihovny jsou dostupné jako open-source \href{https://github.com/RoboticsBrno/BlackBox_library}{zdrojový kód} pod licencí MIT, 
ale také i~jako knihovna v~PlatformIO registry \cite{pio-registry}.