\documentclass{template/socthesis}

\usepackage[czech]{babel}
\usepackage[T1]{fontenc} % evropské uvozovky
\usepackage{csquotes}
\usepackage{xpatch}
\usepackage[author=,status=draft]{fixme} % vkládání poznámek  
% dva módy (status): draft (poznámky se zobrazují v PDF) / final (poznámky se nezobrazují v PDF)

\usepackage{subcaption}
\usepackage[backend=bibtex,bibstyle=numeric,sorting=none,date=long,dateabbrev=false,texencoding=utf8,bibencoding=utf8,style=iso-numeric]{biblatex}
\usepackage{amsmath}
\usepackage{enumitem}

\addbibresource{text.bib}

\titlecz{SDK pro výuku programování na IoT stavebnici}
\titleen{SDK for educational IoT kit}
\author{Tomáš Rohlínek}
\field{18} % Obory SOČ: 1 - 18 (http://www.soc.cz/obory-soc/)
\school{Střední průmyslová škola a Vyšší odborná škola Brno, Sokolská, příspěvková organizace}
\mentor{Vojtěch Boček}
\mentorstatement{Vojtěchu Bočkovi}

% Změňte, pokud se liší
%\region{Jihomoravský}
% \placefooter{Brno 2017}

\begin{document}

\maketitle

\makecopyrightstatement{V~Drásově}

\makethanks{Děkuji svému školiteli Vojtěchu Bočkovi}
\fxfatal{Dopsat poděkování}

\pagestyle{empty}

\section*{Anotace}
Cílem této práce je vytvořit SDK pro práci a výuku na IoT stavebnici BlackBox postavené na platformě ESP32. \fxfatal{odkaz}

\subsection*{Klíčová slova}
BlackBox; IoT; \fxnote{další buzz slovíčka}

\vspace{20mm}

\section*{Annotation}
\fxfatal{překlad}

\subsection*{Keywords}
\fxfatal{překlad}

\newpage
\pagestyle{plain}

\tableofcontents % vysází obsah

%%% Začátek práce
\setcounter{figure}{0}
\setcounter{table}{0}
\newpage

%%% Úvod
\chapter*{Úvod}

V létě roku 2019 jsem vedl robotický tábor a potřeboval jsem narychlo nachystat výrobek, který bychom mohli s účastníky programovat.
Vznikl tak první elektronický trezor později pojmenovaný BlackBox.

Od té doby prošel hardware několika verzemi, od kupky Arduino modulů po komplexní desku.
Jakkoliv dobrý hardware však v dnešní době není nic bez softwaru, který by ho řídil, proto vznikla tato práce.
Jejím cílem je napsat SDK pro desku BlackBox a to pro tři skupiny lidí, organizátory outdoorových her, učitele programování a běžné robotiky.
Nutno podotknout, že se práce nezabývá vytvořením hardwarové části, to je náplní samostatné práce\cite{BlackBox_hardware}.

% \fxnote{"zařízení" není to nejlepší slovo}
% The purpose of this work is to create a~"device" that can be used as both educational development kit and tool for creating complex outdoor games with minimal/automated setup and maximum reusability

% \fxnote{možná použít historii jako odstavec? (jak to vše začalo)}


\newpage
\printbibliography[title=Literatura]
\addcontentsline{toc}{section}{Literatura}

\listoffigures
\addcontentsline{toc}{section}{Seznam obrázků}

\listoftables
\addcontentsline{toc}{section}{Seznam tabulek}

\listoflistedequation
\addcontentsline{toc}{section}{Seznam rovnic}

\end{document}
