\chapter{Úvod}

\enlargethispage{10mm}

V létě roku 2019 jsem vedl robotický tábor a potřeboval jsem narychlo nachystat výrobek, který bychom mohli s účastníky programovat.
Vznikl tak první elektronický trezor později pojmenovaný BlackBox.

Od té doby prošel hardware několika verzemi, od kupky Arduino modulů po komplexní desku.
Jakkoliv dobrý hardware však v dnešní době není nic bez softwaru, který by ho oživoval a řídil, proto vznikla tato práce.
Jejím cílem je napsat SDK pro desku BlackBox a to pro tři skupiny lidí: organizátory outdoorových her, učitele programování a běžné robotiky.

\begin{itemize}[noitemsep]
    \item \textbf{Organizátoři} outdoorových her a zážitkových akcí ocení, když mohou vzít hotové zařízení (BlackBox), jen do něj nahrát program stažený z webu a mít technickou část přípravy své akce hotovou.
    
    \item \textbf{Učitelé programování} budou rádi, že mohou své studenty zaujmout zařízením, které se studenty může přímo 
    interagovat. BlackBox se také speciálně hodí pro výuku programování mikrokontrolérů a hardwarových periferií. 
    
    
    \item  \textbf{Robotici}, hlavně ti pokročilejší, uvítají možnost BlackBox libovolně nakonfigurovat a jednotlivé periferie BlackBoxu (LED kruh, tlaková plocha, akcelerometr, gyroskop, wi-fi, bluetooth a další) využít, jak zrovna potřebují. 
    
\end{itemize}
Nutno podotknout, že se práce nezabývá vytvořením hardwarové části, to je náplní samostatné práce \cite{BlackBox_hardware}.

% \fxnote{"zařízení" není to nejlepší slovo}
% The purpose of this work is to create a~"device" that can be used as both educational development kit and tool for creating complex outdoor games with minimal/automated setup and maximum reusability

% \fxnote{možná použít historii jako odstavec? (jak to vše začalo)}