\chapter*{Úvod}

V létě roku 2019 jsem vedl robotický tábor a potřeboval jsem narychlo nachystat výrobek, který bychom mohli s účastníky programovat.
Vznikl tak první elektronický trezor později pojmenovaný BlackBox.

Od té doby prošel hardware několika verzemi, od kupky Arduino modulů po komplexní desku.
Jakkoliv dobrý hardware však v dnešní době není nic bez softwaru, který by ho oživoval a řídil, proto vznikla tato práce.
Jejím cílem je napsat SDK pro desku BlackBox a to pro tři skupiny lidí: organizátory outdoorových her, učitele programování a běžné robotiky.

Nutno podotknout, že se práce nezabývá vytvořením hardwarové části, to je náplní samostatné práce \cite{BlackBox_hardware}.

% \fxnote{"zařízení" není to nejlepší slovo}
% The purpose of this work is to create a~"device" that can be used as both educational development kit and tool for creating complex outdoor games with minimal/automated setup and maximum reusability

% \fxnote{možná použít historii jako odstavec? (jak to vše začalo)}