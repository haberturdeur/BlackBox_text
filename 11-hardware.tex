\newpage
\section{Hardware}
SDK je psané pro stavebnici Black vytvořenou Tomášem Vavrincem.
Aktuální verze hardware (v1.1) obsahuje následující funkční bloky:

\begin{itemize}[noitemsep]
    \item Hlavní řídící modul
        \begin{itemize}[noitemsep]
            \item ESP32
        \end{itemize}
    \item Uživatelské rozhraní
        \begin{itemize}[noitemsep]
            \item Touchpad
            \item LED kruh
        \end{itemize}
    \item Zámek
        \begin{itemize}[noitemsep]
            \item Motor
            \item Encodér
            \item IR přijímač
            \item IR vysílač
            \item Zámek sériové linky
        \end{itemize}
    \item Senzory prostředí
        \begin{itemize}[noitemsep]
            \item Real Time Clock
            \item Magnetometr
            \item Akcelerometr
            \item Gyroskop
            \item Barometr
        \end{itemize}
\end{itemize}

\newpage
\subsection{Hlavní řídící modul}
Hlavní řídící modul slouží jako výpočetní a řídící centrum celé desky.

\subsubsection{ESP32}
BlackBox používá ESP32-WROVER jako svůj procesor.
Základní informace:
\begin{itemize}
    \item Dual core
    \item 240 MHz
    \item 4MB flash
    \item 8MB PSRAM
    \item WiFi, Bluetooth
\end{itemize}

\subsection{Uživatelské rozhraní}

\subsubsection{Touchpad}
Touchpad je postavený na čipu LDC1614 a jeho alternativách.\footnote{Použitelné jsou pouze alternativy se 4 kanály t.j. ty co mají tvar LDCXX14}
Pro měření stisku se využívá deformace kovové destičky a indukčního měření její vzdálenosti od 4 plošných cívek.
\fxnote{Tady se dá potenciálně napsat spousta věcí :D}

\subsubsection{LED kruh}
Kruh sestává z 60 adresovatelných RGB LED diod typu WS2812B.