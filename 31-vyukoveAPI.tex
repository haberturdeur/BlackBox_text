\chapter{Výukové API}

Výukové API je navržené pro výuku programování od naprostých začátků, proto musí být co nejjednodušší na pochopení a použití.
Zároveň by však neměla omezovat využití všech možností knihovny.
Z toho důvodů používá výukové API pouze jednoduché funkce.
Protože i použití namespace by mohlo pro začátečníky být složité, nejsou tyto funkce součástí žádného namespace, místo toho pro rozlišení používají prefix "bb" (BlackBox).

\lstinputlisting[language=C++, caption=Výukové API]{code/edu.cpp}

\section{Příklady použití}

Pro výuku programování se BlackBox dá použít dvěma způsoby.

\subsection{Běžná výuka}

Běžnou výukou je myšlena výuka programování tak jak běžně probíhá na většině škol, akorát místo práci v zastaralých programovací jazycích v nezajímavém terminálu budou žáci programovat interaktivní hračku, ktera na ně může blikat a jinak s nimi interagovat, což může výrazně vylepšit motivaci k učení se programování.

\subsection{Výuka hrou}

Výuka hrou představuje mix mezi herním a výukovým API, respektive využívá výukovou verzi herního API.

Právě pro výuku programování jsem vymyslel hru pro robotický tábor.
Idea této hry spočívá ve vytvoření sady úloh v nichž se BlackBox se základním softwarem chová jako pomůcka pro jejich řešení.
Například pokud bychom použili kostru "bankovní loupeže" mohl by BlackBox sloužit jako šperhák.

Na začátku tábory by si účastnící zahráli tuto hru.
Časová náročnost jednotlivých stanovišť je úmyslně vypočítána tak, aby nebylo možné v časovém limitu splnit všechna stanoviště, ale aby bylo možné si je všechny vyzkoušet.

Zbytek tábora pak účastníci budou vylepšovat své pomůcky, například, použijeme-li opět příklad šperháku, pak ono vylepšení bude sestávat z automatického řízení šperháku, tato pomůcka výrazně urychlí dokončení stanoviště.
Účastníci mají po celou dobu přístup k jednotlivým stanovištím pro účely testování.

Na konci tábora se opět zahraje tato hra, nyní však se všemi pomůckami, které umožní dokončení hry a tedy výhru.
