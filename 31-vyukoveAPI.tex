\chapter{Výukové API}

Výukové API je navržené pro výuku programování od naprostých začátků, proto musí být co nejjednodušší na pochopení a~použití.
Zároveň by však neměla omezovat využití všech možností knihovny.
Z~toho důvodů používá výukové API pouze jednoduché funkce.
Protože i~použití namespace by mohlo pro začátečníky být složité, nejsou tyto funkce součástí žádného namespace, místo toho pro rozlišení používají prefix "bb" (BlackBox).

\lstinputlisting[language=C++, caption=Výukové API]{code/edu.cpp}

\section{Příklady použití}

Pro výuku programování se BlackBox dá použít dvěma způsoby.

\subsection{Běžná výuka}

Běžnou výukou je myšlena výuka programování tak, jak běžně probíhá na většině škol, pouze místo programování nezajímavé konzolové aplikace budou žáci programovat interaktivní hračku, která na ně může blikat a jinak s~nimi interagovat \footnote{A to případně i onou formou terminálu, uznal by-li to vyučující za vhodné.}, což může výrazně vylepšit motivaci k~učení se programovat.

\subsection{Výuka hrou}

Výuka hrou představuje mix mezi herním a~výukovým API, respektive využívá výukovou (výrazně zjednodušenou) verzi herního API.

Právě pro výuku programování jsem vymyslel hru pro Robotický tábor.
Idea této hry spočívá ve vytvoření sady úloh, v~nichž se BlackBox se základním softwarem chová jako pomůcka pro jejich řešení.

Jednou z úloh může být odemykání virtuálních zámků, kdy BlackBox bude fungovat jako "šperhák".
Na stanovišti budou záda pro BlackBox, do kterých účastník vloží svůj BlackBox, na LED kruhu se budou jeden po druhém objevovat body, kterými musí účastník plynule projet prstem po touchpadu.
Po projetí určitého počtu bodů bude úloha hotova.

Na začátku tábora by si účastnící zahráli tuto hru.
Časová náročnost jednotlivých stanovišť je úmyslně vypočítána tak, aby nebylo možné v~časovém limitu splnit všechna stanoviště, ale aby bylo možné si je všechny vyzkoušet.

Zbytek tábora pak účastníci budou vylepšovat své pomůcky, použijeme-li opět příklad "šperháku", pak ono vylepšení bude sestávat z~automatického řízení "šperháku", namísto fyzického posouvání prstem po touchpadu. Tato pomůcka výrazně urychlí dokončení stanoviště.
Účastníci mají po celou dobu přístup k~jednotlivým stanovištím pro účely testování.

Hra se bude hrát na začátku a na konci tábora, což dává účastníků spoustu času na vylepšování a testování jejich pomůcek.
