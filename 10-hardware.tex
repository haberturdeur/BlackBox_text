\newpage
\chapter{Krátké shrnutí možností hardware}
Můj software je psaný pro stavebnici BlackBox, která byla vytvořená Tomášem Vavrincem.%todo přidat odkaz/literaturu
Aktuální verze hardware (v1.1) obsahuje následující funkční bloky:

\begin{itemize}[noitemsep]
    \item Hlavní řídící modul
        \begin{itemize}[noitemsep]
            \item ESP32 včetně Wi-fi a bluetooth
            \item Real Time Clock
        \end{itemize}
    \item Uživatelské rozhraní
        \begin{itemize}[noitemsep]
            \item Touchpad
            \item LED kruh
        \end{itemize}
    \item Zámek
        \begin{itemize}[noitemsep]
            \item Motor
            \item Enkodér
            \item IR přijímač
            \item IR vysílač
            \item Zámek sériové linky
        \end{itemize}
    \item Senzory prostředí
        \begin{itemize}[noitemsep]
            \item Magnetometr
            \item Akcelerometr
            \item Gyroskop
            \item Barometr
        \end{itemize}
\end{itemize}

\newpage
\section{Hlavní řídící modul}
Hlavní řídící modul slouží jako výpočetní a řídící centrum celé desky.

\subsection{ESP32}
BlackBox používá ESP32-WROVER jako svůj procesor.
Základní informace:
\begin{itemize}
    \item Dual core
    \item 240 MHz
    \item 4MB flash
    \item 8MB PSRAM
    \item WiFi, Bluetooth
\end{itemize}

\subsection{Real Time Clock}
Kvůli snížení spotřeby energie bylo implementováno několik mechanismů.
Jeden z těchto mechanismů je i vypínání všech nepotřebných periferií a uspání ESP32. V takovém případě ale není možné uchovat aktuální čas, proto byl na BlackBox přidán modul RTC, ten je napájen přímo z baterií a není tak závislý na zbytku BlackBoxu.

\section{Uživatelské rozhraní}

\subsection{Touchpad}
Touchpad je postavený na čipu LDC1614 a jeho alternativách.\footnote{Použitelné jsou pouze alternativy se 4 kanály t.j. ty, co mají tvar názvu LDCXX14.}
Pro měření stisku se využívá deformace kovové destičky a indukčního měření její vzdálenosti od 4 plošných cívek.
\fxnote{Tady se dá potenciálně napsat spousta věcí :D}

\subsection{LED kruh}
Kruh sestává z 60 adresovatelných RGB LED diod typu WS2812B.

\section{Zámek}

\subsection{Motor a Encodér}
Zamykací mechanismus je sestaven tak, aby šel BlackBox zasunout do zad trezoru i v zamčeném stavu, obejde se tedy bez kontroly toho, jestli je při zamykání zasunutý nebo ne.

\subsection{IR komunikace}
IR komunikace je zde obsažená hlavně kvůli synchronizaci se zády trezoru, přesněji k identifikaci, do kterých zad je BlackBox zasunut.
Tato identifikace by se potenciálně dala dělat pomocí senzorů prostředí, ale za účelem jednoduchosti a redundance byla zvolena tato možnost.

\subsection{Zámek sériové linky}
Tento mechanismus byl navrhnut za účelem ochrany BlackBox proti neautorizovanému přepsání softwaru.

\section{Senzory prostředí}

\subsection{Senzory polohy}
Tato sada senzorů (akcelerometr, gyroskop, magnetometr) je na různých verzích desky realizována různým způsobem, na verzi 1.0 je realizována jedním čipem obsahujícím všechny tři senzory, ale na verzi 1.1 je realizována pomocí dvou čipů (akcelerometr + gyroskop a magnetometr)\footnote{Verze 1.1 je zpětně kompatibilní, takže se na ni dá osadit i čip z verze 1.0, toho je však v době psaní této práce nedostatek, a proto byl nahrazen na verzi 1.1 dvěma běžnějšími čipy.}.
Knihovna však musí podporovat všechny možnosti.
Tyto senzory zjišťují natočení BlackBoxu v prostoru v 9ti osách.
Do budoucna se chystá rozšíření o GPS senzor. 

\subsection{Barometr}
Barometr zde slouží hlavně k hrubému měření výšky, na které se BlackBox nachází.
Případně se také dá použít jako primitivní způsob předpovědi počasí.